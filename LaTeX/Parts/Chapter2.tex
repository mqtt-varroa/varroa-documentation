\chapter{Concepts}
To understand the workings of Varroa, we will have to take a look at the different parts that make up the system.


\section{Definitions}

\paragraph{Varroa Distributed System}
An orchestration of multiple Varroa instances consisting of one Commander and at least one Agent.

\paragraph{Commander}
The Commander is a part of the Varroa Distributed System, that parses the scenario, generates chunks and distributes them to the Agents.
Only one Commander exists in a Varroa distributed system.

\paragraph{Agent}
The Agent is part of the Varroa Distributed System.
It receives Chunks from the Commander and passes them to its MQTT-Agents.
A Varroa distributed system contains at least one Agent.

\paragraph{Varroa Instance}
A Varroa instance is a running Varroa process in a single JVM, can be either Commander or Agent.

\paragraph{MQTT Agent}
A MQTT Agent is part of an Agent that manages MQTT clients.


\paragraph{MQTT Client}
A MQTT client used to execute the Commands to create load.

\paragraph{Scenario}
A scenario is an abstract representation of a real MQTT-Use-Case.
It defines the topology of all participating MQTT clients and brokers.
The scenario enables the simulation of a large amount of MQTT clients.

\paragraph{Client group}
A group of Clients that share similar behaviour and properties.

\paragraph{Command}
A command is an abstract representation of a work step that must be executed by a MQTT client.
%A command is a particular type of MQTT-Message, for example Connect, Disconnect, Publish or Subscribe.

\paragraph{Chunk}
The scenario is split in Chunks by the Commander and then those Chunk are distributed to the agents.
%A Chunk is a Collection of Information, such as which commands are to be sent to the broker as well as how many clients should perform these commands.
%Another important information, contained in the Chunk is at which rate these commands are to be executed.

%A scenario is a XML-Document which defines a sequence of stages.

\section{General MQTT Concepts}
\paragraph{Broker}
The Broker serves as an intermediary between publishers and subscribers.
It takes over the routing of the exchanged MQTT messages and is the central control authority of a MQTT Network.
%TODO cite Georg
\paragraph{Client}
A client that implements the MQTT protocol. 
\paragraph{Topic}
Topics are strings separated by slashes that do not contain wild-cards.
Messages published with Topic are delivered to subscribers that have registered matching Topic Filters.
\paragraph{Topic Filter}
A Topic Filter is a chain of strings delimited by slashes that 
can cover one or more topics. 
It can also contain wild-cards: a plus sign covers a single hierarchy level, a double cross selects all possible following levels.
\paragraph{Publisher}
Publishers are clients that produce data.
They send messages with a specific topic to the broker.
\paragraph{Subscriber}
Subscribers are clients that subscribe to a subset or to all messages sent via the
MQTT network.
They log on to the broker and register with one or more topic filters that specify the topics whose messages they want to receive.
%TODO cite Georg
\paragraph{Connect}
Disconnect describes the process of a Client establishing a connection to a Broker.
\paragraph{Disconnect}
Connect describes the process of a Client terminating its connection to a Broker.

%\section{Requirements}
%\begin{table}[h]
%	\begin{tabular}{| l | l | p{5cm} | l |}
%		\hline
%		\rowcolor{Gray}
%		\# & Title & User Story & Importance \\
%		\hline
%		1 & Transparency & Varroa has to be comprehensible for the user. & Must have high \\
%		\hline
%		2 & 10.000.000 MQTT Clients & Varroa has to be able to generate a large amount of clients. & Must have high \\
%		\hline
%		3 & Scalability & Varroa should scale vertically with relatively low scaling costs. & Must have \\
%		\hline
%		4 & Determinism & Varroa has to work in deterministic ways, meaning it should produce the same result for a Scenario every time. & Must have \\
%		\hline
%		5 & Distributed & Varroa is a distributed System. & Must have low \\
%		\hline
%		6 & Usabillity & Varroa has to be easily usable. & Very important \\
%		\hline
%		7 & Code Quality & Varroa's coding quality should be very high. & Important\\
%		\hline
%		8 & Stability & Varroa has to run in a stable manner. & Important \\
%		\hline
%		9 & Resource efficiency & Varroa has to use the available computation and memory resources efficiently. & Important\\
%		\hline
%		10 & User / Developer Guide & Varroa needs a User / Developer Guide. & Somewhat important \\
%		\hline
%		11 & Automation capacity & Varroa should be automatable & Somewhat important \\
%		\hline
%	\end{tabular}
%\end{table}