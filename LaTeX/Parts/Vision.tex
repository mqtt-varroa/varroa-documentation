\chapter{Vision}
The lack of testability of MQTT systems was the motivation for creating \emph{Varroa} -- a MQTT testing tool.
The name \emph{Varroa} is inspired by the varroa mite, which is a species of mite that infects honey bee colonies.
Figuratively our MQTT testing tool works in a similar way but instead of infesting a hive, it tries to infest a broker.
This association between the bee hives of the natural realm and those of the MQTT world came from the \emph{HiveMQ} MQTT broker's branding.


The basic use-case of \emph{Varroa} is testing the resilience of brokers by creating load.
To achieve this, \emph{Varroa} is able to simulate a large amount of MQTT clients by a simple and descriptive Scenario definition.
A Scenario can be easily declared by a domain specific language which is defined in a XSD file.


Furthermore Varroa can be executed as a distributed system.
If this is the case the workload is automatically split and distributed to different machines.
This enables great horizontal scaling potential.
In this context scaling means that the amount of MQTT clients can easily be increased.


