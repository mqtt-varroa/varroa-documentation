\chapter{Execution}
To start a Varroa test the User needs to start a Commander and at least one Agent.
The Agents periodically try to establish a connection to the commander.
When all Agents are connected the execution of the scenario starts.

\section{Commander}
\begin{figure}[H]
	\begin{center}
	\includegraphics[scale=0.9]{Resources/PDF/CommanderStates}
	\caption{Commander States}
	\label{pic:CommanderStates}
	\end{center}
\end{figure}
After starting a Varroa instance that is configured as a Commander it waits for the agents in the network to connect.
Until all agents are successfully connected the Commander remains in the 'Waiting for Agents' state.
As the last missing agent connects the Commander switches its state to 'All Agents Connected'.
Then it begins to parse the Scenario's XML file and within finishing it starts to distribute the Scenario's first stage among the agents.
In doing so the Commander changes its state to 'Scenario in Progress'.
It remains in this stage until all stages are executed successfully and then transfers its state into 'Scenario Finished'.
Contrary to this case a faulty execution of a stage results in the failure of the whole Scenario, logically the Commanders state is now 'Scenario Failed'.
Regardless of success or failure the Commander terminates.
To allow better comprehension figure \ref{pic:CommanderStates} illustrates this process.

\subsection{Agent}
\begin{figure}[H]
	\begin{center}
	\includegraphics[scale=0.9]{Resources/PDF/AgentStates}
	\caption{Agent States}
	\label{pic:AgentStates}
	\end{center}
\end{figure}

\section{Configuration}
\subsection{Commander}
\begin{lstlisting}[caption={Commander XML configuration}, captionpos=b, label={lst:commanderConfig}, language=XML]
<varroa>
    <commander>
		<bind-host>192.127.0.1</bind-host>
        <bind-port>12345</bind-port>
        <amount-agents>3</amount-agents>
    </commander>
</varroa>
\end{lstlisting}
\begin{itemize}
	\item \textbf{bin-host:} specifies the Address the commander binds to
	\item \textbf{bind-port:} specifies the port the commander binds to for waiting for Agent connections
	\item \textbf{amount-agents:} specifies the amount of Agents that connect to the Commander
\end{itemize}

\subsection{Agent}
\begin{lstlisting}[caption={Agent XML configuration}, captionpos=b, label={lst:agentConfig}, language=XML]
<varroa>
    <agent>
		<commander-host>192.127.0.1</commander-host>
        <commander-port>12345</commander-port>
        <local-port>23458</local-port>
		<commander-retry-interval>10</commander-retry-interval>
    </agent>
</varroa>
\end{lstlisting}
\begin{itemize}
	\item \textbf{commander-port:} specifies the port of the Commander
	\item \textbf{commander-host:} specifies the Address of the Commander
	\item \textbf{local-port:} specifies the local port the Agent uses for the outgoing connection to the Commander %TODO must be different on every Agent
	\item \textbf{commander-retry-interval:} the time interval in which agents try to connect to the commander
\end{itemize}