\chapter{Vision}
The Name of our MQTT-Testing-Tool (Varroa) is inspired by the varroa mite, which is a species of mite that infects honey bee colonies.
This name has been chosen due to it working in a similar way but instead of infesting a hive, it tries to infest a broker.
The inspiration for this name came from the Broker \enquote{HiveMQ} and it's branding.
The basic use-case of Varroa is testing the resilience of Brokers by creating load.
Hereby load is defined by a number of MQTT-Clients sending different sequences of MQTT-messages to the Broker. 
Which sequences get carried out in which order is determined by a Scenario.
A Scenario is a concept which defines the temporal execution as well as the amount of actions across a MQTT-Network.
The motivation for the creation of this project was the lack of testability of MQTT-Systems without using large amounts of machines.
Therefore Varroa had to be a distributed System itself, to be able to test the Limits of a MQTT-Broker without using a machine for every client.
