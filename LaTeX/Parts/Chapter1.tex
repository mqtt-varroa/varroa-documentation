\chapter{Vision}
The Name of our MQTT-Testing-Tool (Varroa) is inspired by the varroa mite, which is a species of mite that infects honey bee colonies.
This name has been chosen due to it working in a similar way but instead of infesting a hive, it tries to infest a broker.
The inspiration for this name came from the broker \enquote{HiveMQ} and it's branding.
The basic use-case of Varroa is testing the resilience of brokers by creating load.
Hereby load is defined by a number of MQTT-clients sending different sequences of MQTT-messages to the broker. 
Which sequences get carried out in which order is determined by a Scenario.
A scenario defines the temporal execution as well as the amount of actions across a MQTT-network and the topology of the network.
The motivation for the creation of this project was the lack of testability of MQTT-systems.\\
\\
Varroa is organized as a distributed system, due to the impossibility of creating enough MQTT-clients on a single machine to overload a MQTT-broker, especially if the broker is also a distributed system.

