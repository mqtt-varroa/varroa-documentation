\chapter{Scenario Concept}\label{sec:ScenarioConcept}

%\begin{lstlisting}[caption={Implementierung des Trainierens der Markov-Chain}, captionpos=b, label={lst:train}, language=XML]
% <!-- -->
%\end{lstlisting}


\begin{lstlisting}[caption={example for the XML definition of the Scenario}, captionpos=b, label={lst:scenario}, language=XML]
<scenario>
	<broker id="b1">
		<!-- definition broker parameters -->
	</broker>
	
	<clientGroups>
		<!-- definition of client groups -->
	</clientGroups>
	
	<topicGroups>
		<!-- definition of topic groups -->
	</topicGroups>
	
	<subscriptions>
		<!-- definition of subscriptions -->
	</subscriptions>

	<waitPatterns>
		<!-- definition of waitPatterns -->
	</waitPatterns>
	
	<stages>
		<!-- definition of stages -->
	</stages>
</scenario>
\end{lstlisting}

A Varroa test is the execution of a user defined Scenario, which is specified in a XML file.
To define a valid scenario the user needs to define the topology of the Scenario and the behaviour of its components.
An example for the outline of a Scenario XML file is given in \ref{lst:scenario}.

\section{Broker}
\begin{lstlisting}[caption={example for the XML definition of the Broker}, captionpos=b, label={lst:broker}, language=XML]
<scenario>
	<broker id="b1">
		<address>broker.hivemq.com</address>
		<port>1883</port>
	</broker>
</broker>
\end{lstlisting}
The broker is the central component whose performance and stress resistance is tested by the Scenario.
For this the user needs to define:
\begin{itemize}
	\item \textbf{id:} the identifier to reference the broker.
	\item \textbf{address:} the address of the broker can either be a IPv4-Address or a Fully-Qualified-Domain-Name.
	\item \textbf{port:} the port on which the broker is waiting for connections.
\end{itemize}

\section{Client Groups}
\begin{lstlisting}[caption={example for the XML definition of Client Groups}, captionpos=b, label={lst:clientGroups}, language=XML]
<scenario>
	<clientGroups>
		<clientGroup id="cg1">
			<clientIdPattern>A[1-9]+</clientIdPattern>
			<count>100</count>
			<mqtt>
				<!-- MQTT properties -->
			</mqtt>
		</clientGroup>

		<!-- definition of more client groups -->
	</clientGroups>
</scenario>
\end{lstlisting}
A Scenario contains a number of Client Groups.
A Client Group is a specific amount of MQTT clients that share the exact same behaviour.
When defining a Client Group the user needs to specify:
\begin{itemize}
	\item \textbf{id:} the identifier to reference the Client Group.
	\item \textbf{clientIdPattern:} a regular expression to deterministically create individual names for every MQTT client in the Client Group.
	\item \textbf{count:} the amount of MQTT clients contained in the Client Group.
	\item \textbf{mqtt:} MQTT properties of the MQTT clients (see \lstlistingname{} \ref{lst:mqttProperties}).
	\begin{itemize}
		\item \textbf{version (requrired):} the MQTT version this Client Group implements.
		%TODO \item \textbf{cleanStart (optional):} 
		%TODO \item \textbf{sessionExpiryInterval (optional):}
	\end{itemize}
\end{itemize}
\ 
\begin{lstlisting}[caption={example for the XML definition of MQTT properties}, captionpos=b, label={lst:mqttProperties}, language=XML]
<mqtt>
	<version>5</version>
<\mqtt>
\end{lstlisting}
%	<cleanStart>true</cleanStart>
%	<sessionExpiryInterval>0x0</sessionExpiryInterval>
%</mqtt>	
%\end{lstlisting}


\section{Topic Groups} \label{sec:topicGroups}
\begin{lstlisting}[caption={example for the XML definition of Topic Groups}, captionpos=b, label={lst:topicGroups}, language=XML]
<scenario>
	<topicGroups>
		<topicGroup id="tg1">
			<topicNamePattern>topic/subtopic-[0-9]{10}</topicNamePattern>
			<count>10</count>
		</topicGroup>

		<!-- definition of more topic groups -->
	</topicGroups>
</scenario>
\end{lstlisting}
A Scenario contains a number of Topic Groups.
A Topic Group is a specific amount of MQTT Topics whose names are created from the same regular expression.
When defining a Topic Group the user needs to specify:
\begin{itemize}
	\item \textbf{id:} the identifier to reference the Topic Group.
	\item \textbf{topicNamePattern:} a regular expression to deterministically create individual topic names for every member of the Topic Group.
	\item \textbf{count:} the amount of MQTT topics in the topic group.
\end{itemize}

\section{Subscriptions} \label{sec:subscriptions}
\begin{lstlisting}[caption={example for the XML definition of subscriptions}, captionpos=b, label={lst:subscriptions}, language=XML]
<scenario>
	<subscriptions>
		<subscription id="sub-1">
			<topicGroup>tg1</topicGroup>
			<wildCard>false</wildCard>
		</subscription>

		<subscription id="sub-2">
			<topicFilter>/topic/subtopic/subsubtopic/#</topicFilter>
		</subscription>

		<!-- definition of more subscriptions -->
	</subscriptions>
</scenario>
\end{lstlisting}
A Scenario can contain Subscriptions.
A Subscription defines a certain subscription behaviour that Client Groups can implement in their Subscribe Commands.
A Subscription can either target Topics that match a specific Topic Filter or a referenced Topic Group.
To define a Topic Group the user needs to specify:
\begin{itemize}
	\item \textbf{id:} the identifier to reference the Subscription.
	\item \textbf{topicFilter:} the Topic Filter to target specific Topics.
\end{itemize}
or
\begin{itemize}
	\item \textbf{id:} the identifier to reference the Subscription.
	\item \textbf{topicGroup:} reference to a Topic Group.
	\item \textbf{wildCard:} When true the topic name pattern of the Topic Group is analysed and all expandable regex are replaced with single level wildcards ("+"). 
	The regex must contain an expandable topic level separator ("/")
\end{itemize}

\section{Wait Patterns} \label{sec:waitPatterns}
\begin{lstlisting}[caption={example for the XML definition of waitPatterns}, captionpos=b, label={lst:waitPatterns}, language=XML]
<scenario>
	<waitPatterns>
		<waitPattern id="waitPattern-1">
			<subscription>subscription-1</subscription>
			<messagePattern>payload[0-9]{10}</messagePattern>
		</waitPattern>

		<-- definition of more wait patterns -->
	</waitPatterns>
</scenario>
\end{lstlisting}
A Wait Pattern enables the user to model waiting behaviours where MQTT clients waits for receiving the messages of a Subscription (see \ref{sec:subscriptions}) before executing other Commands.
As seen in the example above, the following parameters need to be specified:
\begin{itemize}
	\item \textbf{id:} the identifier to reference the Wait Pattern.
	\item \textbf{subscription:} the Subscription the Wait Pattern targets.
	\item \textbf{messagePattern (optional):} the message pattern that is waited for on the targeted Subscription.
\end{itemize}

\section{Stages and Lifecycles}
\begin{lstlisting}[caption={example for the XML definition of Stages}, captionpos=b, label={lst:stages}, language=XML]
<scenario>
	<stages>
		<stage id="s1" expectedDuration="10s">
			<lifecycle id="s1.l1" clientGroupId="cg1">
				<!-- definition of commands -->
			</lifecycle>
	
			<!-- definition of more lifecycles -->
		</stage>
	
		<!-- definition of more stages -->
	</stages>
</scenario>
\end{lstlisting}
A Scenario is divided into one or more Stages.
These Stages are sequentially executed in the order in which they are specified in the XML document.
A Stage contains a number of Lifecycles, which specify the behaviour of Client Groups in the stage.
Lifecycles within a stage are executed in parallel.
To define a stage the user needs to specify:
\begin{itemize}
	\item \textbf{id:} the identifier to reference the Stage.
	\item \textbf{expectedDuration (optional):} the expected amount of time this Action takes to be executed by the whole Client Group. Used for reporting
\end{itemize}
A Lifecycle contains an amount of Commands (see \ref{sec:commands}) which are assigned to be executed by a certain Client Group.
To define a Lifecycle the user needs to specify:
\begin{itemize}
	\item \textbf{id:} the identifier to reference the Lifecycle.
	\item \textbf{clientGroupId:} the Client Group that executes the Commands in this Lifecycle.
\end{itemize}

\section{Commands}\label{sec:commands}
Commands are executed by Client Groups.
Going into detail this means that the Command is executed by every client in the Client Group.
Commands are contained in Lifecycles, which are assigned to Client Groups.

The structure of Stages, Lifecycles and Commands is described by a small Domain Specific Language which adheres to the following grammar:

\setlength{\grammarparsep}{20pt plus 1pt minus 1pt} % increase separation between rules
\setlength{\grammarindent}{12em} % increase separation between LHS/RHS 
\begin{grammar}
	<stages> ::= <lifecycles>
	\alt <lifecycles><stages>
	
	<lifecycles> ::= <commands>
	\alt <commands><lifecycles>
	
	<commands> := <command>
	\alt <command><commands>
	
	<command> := <connect> \alt <disconnect> \alt <for> \alt <publish> \alt <ramp\_up> \alt <subscribe> \alt <unsubscribe> \alt <wait>
	
	<for> := <for\_header><commands><for\_footer>
	
\end{grammar}

\subsection{connect}
\begin{lstlisting}[caption={example for the XML definition of a connect Command}, captionpos=b, label={lst:connect}, language=XML]
<connect broker="b1" expectedDuration="10s"/>
\end{lstlisting}
The connect command gives an order the Client Group to connect to a broker.
It has the following parameters:
\begin{itemize}
	\item \textbf{broker (required):} the broker the Client Group connects to.
	\item \textbf{expectedDuration (optional):} the expected amount of time this Action takes to be executed by the whole Client Group. Used for reporting.
\end{itemize}

\subsection{disconnect}
\begin{lstlisting}[caption={example for the XML definition of a disconnect Command}, captionpos=b, label={lst:disconnect}, language=XML]
<disconnect/>
\end{lstlisting}
A disconnect Command does not have any parameters.
The Client Group disconnects from the broker it is connected to.

\subsection{publish}
\begin{lstlisting}[caption={example for the XML definition of a publish Commmand}, captionpos=b, label={lst:publish}, language=XML]
<publish topicGroup="tg1" count="10" message="message" rate="100/1s"/>
<publish topicGroup="tg1" count="10" payloadGenerator="randomAlphaNumeric"/>
<publish topicGroup="tg1" message="{{clientId}}" payloadGenerator="mustache"/>
<publish topicGroup="tg1" message="payload[0-9]{10}" payloadGenerator="regex"/>
\end{lstlisting}
A publish Command gives an order the Client Group to publish messages.
It has the following parameters:
\begin{itemize}
	\item \textbf{topicGroup (required):} references the Topic Group the message is published to (see \ref{sec:topicGroups}). 
	\item \textbf{message (optional):} the argument for the payloadGenerator.
	\item \textbf{payloadGenerator (optional):} the generator for the payload that is sent within the message. The following options exist:
		\begin{itemize}
			\item \textbf{randomAlphaNumeric:} generates random alpha numeric strings as payloads.
			\item \textbf{static (default):} simply uses the message parameter as payload.
			\item \textbf{mustache:} generates payloads based on a mustache.js template given in the message parameter.
			\item \textbf{regex:} generates payloads based on a regex pattern given in the message parameter.
		\end{itemize}
	\item \textbf{qos (optional):} the quality of service.
	\item \textbf{waitForACK (optional):} whether the Client Group waits for the ACK from the broker.
	\item \textbf{count (required when a rate is given):} the amount of publishes for every client in the Client Group to execute.
	\item \textbf{rate (optional):} the rate at witch the publishes are executed.
	\item \textbf{expectedDuration (optional):} the expected amount of time this action takes to be executed by the whole Client Group. Used for reporting.
\end{itemize}

\subsection{subscribe}
\begin{lstlisting}[caption={example for the XML definition of a subscibe Command}, captionpos=b, label={lst:subscirbe}, language=XML]
<subscribe subscription="sub1" qos="1"/>
\end{lstlisting}
A subscribe Command gives an order the Client Group to subscribe.
It has the following parameters:
\begin{itemize}
	\item \textbf{subscription (required):} references the Subscription to be executed (see \ref{sec:subscriptions}).
	\item \textbf{qos (optional):} the quality of service.
	\item \textbf{expectedDuration (optional):} the expected amount of time this action takes to be executed by the whole Client Group. Used for reporting.
\end{itemize}

\subsection{unsubscribe}
\begin{lstlisting}[caption={example for the XML definition of a unsubscibe Command}, captionpos=b, label={lst:unsubscirbe}, language=XML]
<unsubscribe subscription="sub1"/>
\end{lstlisting}
A unsubscribe Command gives an order the Client Group to unsubscribe.
It has the following parameters:
\begin{itemize}
	\item \textbf{subscription (required):} references the Subscription to be terminated (see \ref{sec:subscriptions}).
	\item \textbf{expectedDuration (optional):} the expected amount of time this action takes to be executed by the whole Client Group. Used for reporting.
\end{itemize}

\subsection{wait}
\begin{lstlisting}[caption={example for the XML definition of a wait Command}, captionpos=b, label={lst:wait}, language=XML]
<wait waitPattern="wp1" duration="10s"/>
\end{lstlisting}
A wait Command gives an order a Client Group to wait before executing the following Commands.
It has the following parameter:
\begin{itemize}
	\item \textbf{waitPattern (optional):} references the Wait Pattern (see \ref{sec:waitPatterns}) to waited for.
	\item \textbf{duration (optional):} the amount of time to wait.
	\item \textbf{expectedDuration: (optional):} the expected amount of time this action takes to be executed by the whole Client Group. Used for reporting.
\end{itemize}
If both a Wait Pattern and a duration are given the following Commands are executed after either the duration is exceeded or the Wait Pattern has matched.

\subsection{for}
\begin{lstlisting}[caption={example for the XML definition of a for Command}, captionpos=b, label={lst:for}, language=XML]
<for times="2">
	<connect broker="b1"/>
	<disconnect/>
</for>
\end{lstlisting}
A for Command gives an order the Client Group to execute the inner Commands multiple times.
It has the following parameters:
\begin{itemize}
	\item \textbf{times:} the amount of times the inner Commands are executed.
	\item \textbf{rate (optional):} the rate at which the inner Commands are executed.
	\item \textbf{expectedDuration: (optional):} the expected amount of time this action takes to be executed by the whole Client Group. Used for reporting.
\end{itemize}

\subsection{rampUp}
\begin{lstlisting}[caption={example for the XML definition of a rampUp Command}, captionpos=b, label={lst:rampUp}, language=XML]
<rampUp duration="10s"/>	
\end{lstlisting}
The Client Group executes the following Commands of the lifecycle in a staggered manner.
\begin{itemize}
	\item \textbf{duration:} the amount of time the rampUp takes.
\end{itemize}


