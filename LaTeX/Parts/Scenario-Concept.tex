\chapter{Scenario Concept}\label{sec:ScenarioConcept}\vspace{-20pt}
A Varroa test is the execution of a user defined Scenario, which is specified in a XML file.
To define a valid scenario the user needs to declare the topology of the Scenario and the behaviour of its components.
A formal definition of the Domain Specific Language for a Scenario is given in Appendix \ref{sec:ScenarioDomainSpecificLanguage}.
An example for the outline of a Scenario XML file is shown in Listing \ref{lst:scenario}.
\begin{lstlisting}[caption={Example for the XML definition of the Scenario}, captionpos=b, label={lst:scenario}, language=XML]
<scenario>
	<broker id="b1">
		<!-- definition broker parameters -->
	</broker>
	
	<clientGroups>
		<!-- definition of client groups -->
	</clientGroups>
	
	<topicGroups>
		<!-- definition of topic groups -->
	</topicGroups>
	
	<subscriptions>
		<!-- definition of subscriptions -->
	</subscriptions>

	<waitPatterns>
		<!-- definition of waitPatterns -->
	</waitPatterns>
	
	<stages>
		<!-- definition of stages -->
	</stages>
</scenario>
\end{lstlisting}

\section{Brokers}
\begin{lstlisting}[caption={Example for the XML definition of the Broker}, captionpos=b, label={lst:broker}, language=XML]
<brokers>
	<broker id="b1">
		<address>broker.hivemq.com</address>
		<port>1883</port>
	</broker>
</brokers>
\end{lstlisting}
The brokers are the central components whose performances and stress resistances are tested by the Scenario.
To define a broker the user needs to specify:
\begin{itemize}
	\item \textbf{id:} the identifier to reference the broker.
	\item \textbf{address:} the address of the broker can either be a IP-Address or a Fully-Qualified-Domain-Name.
	\item \textbf{port:} the port on which the broker is waiting for connections.
\end{itemize}

\section{Client Groups}
\begin{lstlisting}[caption={Example for the XML definition of Client Groups}, captionpos=b, label={lst:clientGroups}, language=XML]
<scenario>
	<clientGroups>
		<clientGroup id="cg1">
			<clientIdPattern>A[1-9]+</clientIdPattern>
			<count>100</count>
			<mqtt>
				<version>5</version>
			</mqtt>
		</clientGroup>

		<!-- definition of more client groups -->
	</clientGroups>
</scenario>
\end{lstlisting}
A Scenario contains a number of Client Groups.
A Client Group is a specific amount of MQTT clients that share the exact same behaviour.
When defining a Client Group the user needs to specify:
\begin{itemize}
	\item \textbf{id:} the identifier to reference the Client Group.
	\item \textbf{clientIdPattern:} a regular expression to deterministically create individual names for every MQTT client in the Client Group.
	\item \textbf{count:} the amount of MQTT clients contained in the Client Group.
	\item \textbf{mqtt:} MQTT properties of the MQTT clients in the Client Group.
	\begin{itemize}
		\item \textbf{version:} the MQTT version of the MQTT clients in the Client Group.
		%TODO \item \textbf{cleanStart (optional):} 
		%TODO \item \textbf{sessionExpiryInterval (optional):}
	\end{itemize}
\end{itemize}

%\begin{lstlisting}[caption={Example for the XML definition of MQTT properties}, captionpos=b, %label={lst:mqttProperties}, language=XML]
%<mqtt>
%	<version>5</version>
%<\mqtt>
%\end{lstlisting}
%	<cleanStart>true</cleanStart>
%	<sessionExpiryInterval>0x0</sessionExpiryInterval>
%</mqtt>	
%\end{lstlisting}

\section{Topic Groups} \label{sec:topicGroups}
\begin{lstlisting}[caption={Example for the XML definition of Topic Groups}, captionpos=b, label={lst:topicGroups}, language=XML]
<scenario>
	<topicGroups>
		<topicGroup id="tg1">
			<topicNamePattern>topic/subtopic-[0-9]{10}</topicNamePattern>
			<count>10</count>
		</topicGroup>

		<!-- definition of more topic groups -->
	</topicGroups>
</scenario>
\end{lstlisting}
A Scenario contains a number of Topic Groups.
A Topic Group is a specific amount of MQTT Topics whose names are created from the same regular expression.
When defining a Topic Group the user needs to specify:
\begin{itemize}
	\item \textbf{id:} the identifier to reference the Topic Group.
	\item \textbf{topicNamePattern:} a regular expression to deterministically create individual Topic names for every Topic of the Topic Group.
	\item \textbf{count:} the amount of Topics in the Topic Group.
\end{itemize}

\section{Subscriptions} \label{sec:subscriptions}
\begin{lstlisting}[caption={Example for the XML definition of Subscriptions}, captionpos=b, label={lst:subscriptions}, language=XML]
<scenario>
	<subscriptions>
		<subscription id="sub-1">
			<topicGroup>tg1</topicGroup>
			<wildCard>false</wildCard>
		</subscription>

		<subscription id="sub-2">
			<topicFilter>/topic/subtopic/subsubtopic/#</topicFilter>
		</subscription>

		<!-- definition of more subscriptions -->
	</subscriptions>
</scenario>
\end{lstlisting}
A Scenario can contain Subscriptions.
A Subscription defines a certain subscription behaviour that Client Groups can implement in their Subscribe Commands.
A Subscription can either target Topics that match a specific Topic Filter or a referenced Topic Group.
To define a Topic Group the user needs to specify:
\begin{itemize}
	\item \textbf{id:} the identifier to reference the Subscription.
\end{itemize}
and either
\begin{itemize}
	\item \textbf{topicFilter:} the Topic Filter to target specific Topics.
\end{itemize}
or instead
\begin{itemize}
	\item \textbf{topicGroup:} reference to a Topic Group.
	\item \textbf{wildCard (default=false):} when true, the Subscription matches all Topics that are part of the referenced Topic Group.
	This is done by analysing the Topic name pattern of the referenced Topic Group and replacing all expandable regex with MQTT single level wildcards ("+")
	(therefore the regex must not contain an expandable topic level separator ("/")).
	When false, the Topics of the referenced Topic Group are partitioned over all MQTT clients of a Client Group that uses the Subscription.
\end{itemize}

\section{Wait Patterns} \label{sec:waitPatterns}
\begin{lstlisting}[caption={Example for the XML definition of Wait Patterns}, captionpos=b, label={lst:waitPatterns}, language=XML]
<scenario>
	<waitPatterns>
		<waitPattern id="waitPattern-1">
			<subscription>subscription-1</subscription>
			<messagePattern>payload[0-9]{10}</messagePattern>
		</waitPattern>

		<!-- definition of more wait patterns -->
	</waitPatterns>
</scenario>
\end{lstlisting}
A Wait Pattern enables the user to model waiting behaviours where MQTT clients expect to receive messages of a Subscription (see \ref{sec:subscriptions}) before executing other Commands.
As seen in the example above, the following parameters need to be specified:
\begin{itemize}
	\item \textbf{id:} the identifier to reference the Wait Pattern.
	\item \textbf{subscription:} the Subscription the Wait Pattern targets.
	\item \textbf{messagePattern (optional):} the message pattern that is waited for on the targeted Subscription.
\end{itemize}
\newpage
\section{Stages and Lifecycles}
\begin{lstlisting}[caption={Example for the XML definition of Stages}, captionpos=b, label={lst:stages}, language=XML]
<scenario>
	<stages>
		<stage id="s1" expectedDuration="10s">
			<lifeCycle id="s1.l1" clientGroupId="cg1">
				<!-- definition of commands -->
			</lifeCycle>
	
			<!-- definition of more lifecycles -->
		</stage>
	
		<!-- definition of more stages -->
	</stages>
</scenario>
\end{lstlisting}
A Scenario is divided into one or more Stages.
These Stages are sequentially executed in the order in which they are specified in the XML document.
A Stage contains a number of Lifecycles, which specify the behaviour of Client Groups in the Stage.
Lifecycles within a Stage are executed in parallel.
To define a stage the user needs to specify:
\begin{itemize}
	\item \textbf{id:} the identifier to reference the Stage.
	\item \textbf{expectedDuration (optional):} the expected amount of time the Stage takes to be executed. This is useful to specify timing requirements and is used for reporting.
\end{itemize}
A Lifecycle contains an amount of Commands (see \ref{sec:commands}) which are assigned to be executed by a certain Client Group.
To define a Lifecycle the user needs to specify:
\begin{itemize}
	\item \textbf{id:} the identifier to reference the Lifecycle.
	\item \textbf{clientGroupId:} the Client Group that executes the Commands in this Lifecycle.
	\item \textbf{expectedDuration (optional):} the expected amount of time the Lifecycle takes to be executed. This is similar to the expected duration of a Stage but only for a specific Lifecycle.
\end{itemize}
\newpage
\section{Commands}\label{sec:commands}
Commands are executed per Client Group.
Going into detail this means that a Command is executed by every MQTT client in the Client Group.
Commands are contained in Lifecycles, which are assigned to Client Groups.

\subsection{Connect}
\begin{lstlisting}[caption={Example for the XML definition of a Connect Command}, captionpos=b, label={lst:connect}, language=XML]
<connect broker="b1" expectedDuration="10s"/>
\end{lstlisting}
The Connect Command will connect every MQTT client of the Client Group to a broker.
It has the following parameters:
\begin{itemize}
	\item \textbf{broker (optional):} the brokers the Client Group connects to. The identifiers of multiple brokers are seperated by whitespaces.
	If no broker is specified all brokers that are defined in the scenario are connected to. 
	\item \textbf{expectedDuration (optional):} the expected amount of time this Action takes to be executed. Used for reporting.
	\item \textbf{retries (optional): the amout of connection retries.}
	\item \textbf{retryInterval (optional):} the amout of time between retry attempts. 
\end{itemize}

\subsection{Disconnect}
\begin{lstlisting}[caption={Example for the XML definition of a Disconnect Command}, captionpos=b, label={lst:disconnect}, language=XML]
<disconnect/>
\end{lstlisting}
The Client Group disconnects from the broker it is connected to.
The Scenario is validated upfront to only allow sane combinations of Connect and Disconnect Commands.
The Disconnect Command only has the following parameter:
\begin{itemize}
	\item \textbf{expectedDuration (optional):} the expected amount of time this Action takes to be executed. Used for reporting.
\end{itemize}

\subsection{Publish}
\begin{lstlisting}[caption={Example for the XML definition of a Publish Commmand}, captionpos=b, label={lst:publish}, language=XML]
<publish topicGroup="tg1" count="10" message="message" rate="100/1s" retain="true"/>
<publish topicGroup="tg1" message="10" payloadGenerator="randomAlphaNumeric"/>
<publish topicGroup="tg1" message="{{clientId}}" payloadGenerator="mustache"/>
<publish topicGroup="tg1" message="payload[0-9]{10}" payloadGenerator="regex"/>
\end{lstlisting}
A Publish Command gives an order to the Client Group to publish messages.
It has the following parameters:
\begin{itemize}
	\item \textbf{topicGroup:} references the Topic Group the messages are published to (see \ref{sec:topicGroups}). 
	\item \textbf{message:} the argument for the payloadGenerator.
	\item \textbf{payloadGenerator (optional):} the generator for the payload that is sent within the message. The following options exist:
		\begin{itemize}
			\item \textbf{randomAlphaNumeric:} generates random alpha numeric strings as payloads.
			\item \textbf{static (default):} simply uses the message parameter as payload.
			\item \textbf{mustache:} generates payloads based on a mustache.js template given in the message parameter.
			\item \textbf{regex:} generates payloads based on a regex pattern given in the message parameter.
		\end{itemize}
	\item \textbf{qos (default=0):} the MQTT Quality of Service level the messages are published with. Possible values: 0, 1 and 2
	\item \textbf{waitForAck (default=false):} whether the Client Group waits for the acknowledgement of the message from the broker before continuing to publish the next message.
	\item \textbf{count (default=1):} the amount of messages every MQTT client in the Client Group publishes.
	\item \textbf{rate (optional):} the rate at witch the MQTT clients publish the messages. If waitForAck is true and waiting for the acknowledgement exceeds a rate interval, the next message is published immediately after the previous one is acknowledged.
	\item \textbf{expectedDuration (optional):} the expected amount of time this Action takes to be executed. Used for reporting.
	\item \textbf{retain (default=false):} a flag to indicate that the PUBLISH messages are to be sent as retained messages. Used for reporting.
\end{itemize}
\newpage
\subsection{Subscribe}
\begin{lstlisting}[caption={Example for the XML definition of a Subscibe Command}, captionpos=b, label={lst:subscirbe}, language=XML]
<subscribe subscription="sub1 sub2" qos="1"/>
\end{lstlisting}
A Subscribe Command gives an order to the Client Group to subscribe.
It has the following parameters:
\begin{itemize}
	\item \textbf{subscription:} references the Subscriptions to be executed (see \ref{sec:subscriptions}). 
	Note that multiple Subscriptions can be referenced, those need to be separeted by a whitespace.
	\item \textbf{qos (default=2):} the maximum MQTT Quality of Service level with which the matching publish messages are consumed.
	\item \textbf{expectedDuration (optional):} the expected amount of time this Action takes to be executed. Used for reporting.
\end{itemize}

\subsection{Unsubscribe}
\begin{lstlisting}[caption={Example for the XML definition of a Unsubscibe Command}, captionpos=b, label={lst:unsubscirbe}, language=XML]
<unsubscribe subscription="sub1"/>
\end{lstlisting}
A Unsubscribe Command gives an order to the Client Group to unsubscribe.
It has the following parameters:
\begin{itemize}
	\item \textbf{subscription:} references the Subscriptions to be terminated (see \ref{sec:subscriptions}).
	Note that multiple Subscriptions can be referenced, those need to be separeted by a whitespace.
	\item \textbf{expectedDuration (optional):} the expected amount of time this Action takes to be executed. Used for reporting.
\end{itemize}

\subsection{Wait}
\begin{lstlisting}[caption={Example for the XML definition of a Wait Command}, captionpos=b, label={lst:wait}, language=XML]
<wait waitPattern="wp1" duration="10s"/>
\end{lstlisting}
A Wait Command gives an order to the Client Group to wait before executing the following Commands.
It has the following parameters:
\begin{itemize}
	\item \textbf{waitPattern (optional):} references the Wait Pattern (see \ref{sec:waitPatterns}) for messages the MQTT clients expect to receive.
	\item \textbf{duration (optional):} the amount of time to wait.
	\item \textbf{expectedDuration: (optional):} the expected amount of time this Action takes to be executed. Used for reporting.
\end{itemize}
If both a Wait Pattern and a duration are given, the following Commands are executed after either the Wait Pattern has matched or the duration is exceeded.

\subsection{For}
\begin{lstlisting}[caption={Example for the XML definition of a For Command}, captionpos=b, label={lst:for}, language=XML]
<for times="2">
	<connect broker="b1"/>
	<disconnect/>
</for>
\end{lstlisting}
A For Command gives an order to the Client Group to execute the inner Commands multiple times.
It has the following parameters:
\begin{itemize}
	\item \textbf{times:} the amount of times the inner Commands are executed.
	\item \textbf{rate (optional):} the rate at which the inner Commands are executed.
	If the inner Commands take longer to execute than a rate interval, the next iteration is executed immediately after the previous one finished.
	\item \textbf{expectedDuration: (optional):} the expected amount of time this Action takes to be executed. Used for reporting.
\end{itemize}

\subsection{RampUp}
\begin{lstlisting}[caption={Example for the XML definition of a RampUp Command}, captionpos=b, label={lst:rampUp}, language=XML]
<rampUp duration="10s"/>	
\end{lstlisting}
\vspace{-5pt}
The Client Group executes Commands following a RampUp Command in a staggered manner.
\begin{itemize}
	\item \textbf{duration:} the amount of time the rampUp takes.
\end{itemize}


