\chapter{Concepts}
To understand the workings of Varroa, we will have to take a look at the different parts that make up the system.
Furthermore we explain the basic concepts of MQTT.

\section{Varroa Distributed System Concepts}
A Varroa Distributed System is an orchestration of multiple Varroa Instances consisting of one Commander and at least one Agent.
Varroa is organized as a distributed system due to the impossibility of creating enough MQTT clients on a single machine to overload an MQTT broker, especially if the broker is also a distributed system.

\paragraph{Varroa Instance}
A Varroa instance is a running Varroa process in a single JVM, which can be a Commander and/or an Agent.

\paragraph{Commander}
The Commander is a part of the Varroa Distributed system that processes the Scenario.
It distributes the workload of the Scenario to the Agents through Chunks.
Only one Commander exists in a Varroa Distributed System.

\paragraph{Agent}
Agents are part of the Varroa Distributed System.
They are responsible for executing the Scenarios workload by receiving Chunks from the Commander and passing them to their MQTT Agents.
A Varroa Distributed System contains at least one Agent.

\paragraph{MQTT Agent}
MQTT Agents are components of an Agent.
Every MQTT Agent manages one MQTT client.

\paragraph{Chunk}
The Scenario is split in Chunks by the Commander and then those Chunks are distributed to the Agents.

\section{Varroa Scenario Concepts}
The integral idea of Varroa is testing Scenarios, which means simulating the behaviour of a large amount of MQTT clients.
It tests whether the MQTT broker can handle the load associated with these clients.
A Scenario is an abstract representation of a real MQTT use case.
It defines the topology of all participating MQTT clients and brokers.

\paragraph{Client Group}
Client Groups are a part of the Scenario.
They enable the user to define an abstract group of MQTT clients with similar behaviour and properties.

\paragraph{Topic Group}
A Topic Group represents a number of topics that share a naming pattern.
This concept enables the user to model the interaction between Client Groups and a number of similar topics.

\paragraph{Stage}
Stages are execution steps of the Scenario. Stages are executed serially.

\paragraph{Lifecycle}
Lifecycles are part of a Stage.
One Lifecycle defines the behaviour of one Client Group within the Stage.
The Lifecycles contained in a Stage are executed in parallel.

\paragraph{Command}
A Lifecycle consists of multiple Commands.
A Command is an abstract representation of a work step that must be executed by a MQTT Agent.

\paragraph{Action}
An Action is the execution of a Command by a Client Group.

\section{General MQTT Concepts}

\paragraph{MQTT Broker}
The broker serves as an intermediary between publishers and subscribers.
It takes over the routing of the exchanged MQTT messages and is the central control authority of a MQTT network.
%TODO cite Georg

\paragraph{MQTT Client}
A client that implements the MQTT protocol.
We use the \emph{HiveMQ MQTT Client} as an implementation.

\paragraph{Topic}
Topics are strings separated by slashes that do not contain wildcards.
Messages published with a topic are delivered to subscribers that have registered matching Topic Filters.

\paragraph{Topic Filter}
A Topic Filter is a chain of strings delimited by slashes that 
can cover one or more topics. 
It can also contain wildcards: a single plus character covers one hierarchy level, a double cross selects all possible following levels.

\paragraph{Publish}
Publishers are clients that produce data.
They send messages with a specific Topic to the broker.

\paragraph{Subscribe}
Subscribers are clients that subscribe to a subset or to all messages sent via the
MQTT network.
They log on to the broker and register with one or more Topic Filters that specify the topics, which determine what messages they want to receive.
%TODO cite Georg



